\begin{enumerate}
\item[a)] La rutina {\tt mmu_inicializar} en {\tt mmu.c} inicializa en 0 un contador de páginas ocupadas p\'aginas ocupadas, que se ir\'a incrementando en la medida en que se utilicen las p\'aginas, y llama a la funci\'on {\tt mmu_inicializar_dir_kernel}, previamente creada.\\
Contamos también con la función {\tt obtener_pagina_libre} que se encarga de incrementar el contador de páginas ocupadas y retornar el valor de una página libre reservada. Para tratar de evitar problemas de solapamiento, resolvimos incrementar de a 2 el contador de páginas ocupadas, lo que terminaría en la reserva de 2 páginas. TAL VEZ HAY QUE CAMBIAR ESTO.

\item[b)] Se escribe la rutina {\tt mmu_inic_dir_pirata} para inicializar un directorio de páginas y tablas de páginas para una tarea. Esta función solo se encargará de inicializar el directorio de páginas y las tablas para un tarea, en lugar de tener que también copiar el código de la misma, responsabilidades que dejamos para otras funciones. Para el trabajo de copiar el c\'odigo del zombie y mapear las p\'aginas correspondientes, se crean las funciones {\tt copiar\_código} y {\tt tarea\_al\_mapa} en {\tt mmu.c}. Su funcionamiento resulta muy similar al de inicializar el directorio y las tablas para el kernel. Difiere en el hecho de que se debe pedir una página libre para el directorio y las tablas de páginas, devolviendo la dirección de la página pedida para el directorio.\\

