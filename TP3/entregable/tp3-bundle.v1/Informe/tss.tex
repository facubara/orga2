\begin{enumerate}

\item[a)]Definimos las entradas 13 (tarea inicial), 14 (tarea idle), 15-22 (tareas jugador 1), 23-30 (tareas jugador 2) de la GDT
A estas entradas se les asigna tipo 9 (Execute-Only,accessed), base 0, presente 1 y DPL 0. El límite se establece en {\tt 0x68} para que sea mayor al tamaño de una TSS. La Figura \ref{fig:gdt2} muestra el estado de la GDT luego de agregar estas entradas.

\item[b)]Para completar la TSS de la tarea Idle hacemos uso de la función {\tt tss\_inicializar\_tarea\_idle}, la cual se encarga de asignar los segmentos de datos del kernel a los campos GS, FS, DS, SS, ES y poner el segmento de código de kernel en el campo CS. A su vez, completa ESP y EBP con la dirección del stack del kernel y se ocupa de que comparta el cr3 con el kernel. Por último, el EIP será {\tt 0x16000}, el campo EFLAGS se completa con {\tt 0x202} (para habilitar las interrupciones), el IOMAP con {\tt 0xFFFF} y el resto se deja en 0.

\item[c)] La función {\tt tss\_inicializar\_tareas\_piratas} toma como parámetro un puntero a tss y se ocupa de llenar sus campos. Los campos ES, SS, DS, FS y GS se completan con el segmento de datos de usuario, y el campo CS se completa con el segmento de código de usuario. Tanto ESP como EBP quedan seteados en {\tt 0x401000}, el campo EFLAGS en {\tt 0x202}, el CR3 en 0 (se le asignará un cr3 al momento de lanzar el pirata correspondiente utilizando la función {\tt mmu_inicializar_dir_tarea}), se pedirá una página nueva para el ESP0, SS0 se completa con segmento de datos kernel, IOMAP con {\tt 0xffff} y el resto con 0. 

\item[d)] En la rutina {\tt tss\_inicializar} pedimos una página libre para la tss de la tarea inical y asignamos la dirección obtenida en el campo {\it base} de la entrada correspondiente a esta tarea en la GDT.

\item[e)] En la rutina {\tt tss\_inicializar} completamos la entrada de la GDT perteneciente a la tarea Idle, completando el campo {\it base} con la dirección correspondiente a la misma.

\item[f)]

\item[g)]

\item[h)]
