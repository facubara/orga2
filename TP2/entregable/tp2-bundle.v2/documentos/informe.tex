\documentclass[a4paper]{article}
\usepackage[spanish]{babel}
\usepackage[utf8]{inputenc}
\usepackage{charter}   % tipografia
\usepackage{graphicx}
%\usepackage{makeidx}
\usepackage{paralist} %itemize inline

%\usepackage{float}
%\usepackage{amsmath, amsthm, amssymb}
%\usepackage{amsfonts}
%\usepackage{sectsty}
%\usepackage{charter}
%\usepackage{wrapfig}
%\usepackage{listings}
%\lstset{language=C}


\input{codesnippet}
\input{page.layout}
% \setcounter{secnumdepth}{2}
\usepackage{underscore}
\usepackage{caratula}
\usepackage{url}


% ******************************************************** %
%              TEMPLATE DE INFORME ORGA2 v0.1              %
% ******************************************************** %
% ******************************************************** %
%                                                          %
% ALGUNOS PAQUETES REQUERIDOS (EN UBUNTU):                 %
% ========================================
%                                                          %
% texlive-latex-base                                       %
% texlive-latex-recommended                                %
% texlive-fonts-recommended                                %
% texlive-latex-extra?                                     %
% texlive-lang-spanish (en ubuntu 13.10)                   %
% ******************************************************** %



\begin{document}


\thispagestyle{empty}
\materia{Organización del Computador II}
\submateria{Primer Cuatrimestre de 2015}
\titulo{Trabajo Práctico II}
\subtitulo{SIMD}
\integrante{Alejandro Mignanelli}{609/11}{minga_titere@hotmail.com}
\integrante{Facuuuuu}{XXX/xx}{chabooooon@hotmail.com}
\integrante{Iaaaaaaan}{XXX/xx}{me_la_super_como@gmail.com}

\maketitle
\newpage

\thispagestyle{empty}
\vfill
\begin{abstract}
En el presente trabajo se describe la problemática de procesar información de manera eficiente cuando los mismos requieren:
\begin{enumerate}
\item Transferir grandes volúmenes de datos.
\item Realizar las mismas instrucciones sobre un set de datos importante.
\end{enumerate}
\end{abstract}
\thispagestyle{empty}
\vspace{3cm}
\tableofcontents
\newpage

%\normalsize
\section{Objetivos generales}


\newpage
\section{Preámbulo}

\subsection{Calidad de las Mediciones}



\newpage

\section{Experimentación}


\newpage

\section{Blur}

\subsection{Diferencias de performance en Blur}


\subsubsection{Resultados}


\subsubsection{Conclusiones}


\newpage
\section{Merge}

\subsection{Idea general del algoritmo}



\subsection{Diferencias de performance en Merge}


\subsubsection{Resultados}

\subsubsection{Conclusiones}


\newpage

\section{HSL}

\subsection{Idea general del algoritmo}



\subsection{Diferencias de performance en HSL}



\subsubsection{Resultados}

\subsubsection{Conclusiones}


\newpage
\section{Conclusiones y trabajo futuro}


\end{document}

